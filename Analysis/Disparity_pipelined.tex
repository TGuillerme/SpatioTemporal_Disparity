\documentclass{article}
\usepackage[sc]{mathpazo}
\usepackage[T1]{fontenc}
\usepackage{geometry}
\usepackage{amsmath}
\pagenumbering{arabic}
\geometry{verbose,tmargin=2.5cm,bmargin=2.5cm,lmargin=2.5cm,rmargin=2.5cm}
\setcounter{secnumdepth}{2}
\setcounter{tocdepth}{2}
\usepackage[unicode=true,pdfusetitle,
 bookmarks=true,bookmarksnumbered=true,bookmarksopen=true,bookmarksopenlevel=2,
 breaklinks=false,pdfborder={0 0 1},backref=false,colorlinks=false]
 {hyperref}
\hypersetup{
 pdfstartview={XYZ null null 1}}
\usepackage{Sweave}
\begin{document}
\begin{Schunk}
\begin{Sinput}
> library(knitr)
> # set global chunk options
> opts_chunk$set(fig.path='figure/minimal-', fig.align='center', fig.show='hold')
> options(formatR.arrow=TRUE,width=90)
\end{Sinput}
\end{Schunk}


\title{STD pipelined analysis v2.0!}

\author{Thomas Guillerme}

\maketitle

New version of the spatio-temporal disparity analysis in mammals around the K-Pg boundary (using Beck's and Halliday's data)

\tableofcontents

\section{Before starting}

\subsection{Loading the package}
First we need to install the packages.
Note that if the packages need to be installed manually (the following code is not evaluated, assuming you have these packages already installed).
To download them, just copy/paste the following code.

%TG: Remember to change the Claddis INSTAL PATH whenever (if) Graeme uploads my new functionalities
\begin{Schunk}
\begin{Sinput}
> # The latest version of devtools and ape
> if(!require(devtools)) install.packages("devtools")
> if(!require(ape)) install.packages("ape")
> ## The latest version of Claddis (on development on my own branch)
> install_github("TGuillerme/Claddis")
> ## The latest version of dispRity 
> install_github("TGuillerme/dispRity")
\end{Sinput}
\end{Schunk}


\subsection{Loading the data}

Using the data from Beck & Lee 2014 (same as previous one) as well as Halliday & Goswami 2015.
Halliday's data contains more trees so a next version of the analysis will use the whole tree distribution for both data sets (\texttt{mulTree} style).
The FADLAD tables are downloaded from the paleoDataBase with each first and last occurence datum being the most inclusive (i.e. taking the maximum and minimum time span of the genus).

\begin{Schunk}
\begin{Sinput}
> set.seed(123)
> ## Loading the packages
> library(Claddis) ; library(dispRity)
> ## Loading the discrete morphological matrix
> matrix_Beck <- ReadMorphNexus("../Data/2014-Beck-ProcB-matrix-morpho.nex")
> matrix_Hall <- ReadMorphNexus("../Data/2015-Halliday-LinSoc-matrix-morpho.nex")
> ## Loading the tree
> tree_Beck <- read.nexus("../Data/2014-Beck-ProcB-TEM.tre")
> tree_Hall <- read.nexus("../Data/2015-Halliday-LinSoc-tree.nex")[[1]]
> ## Arbitrarily solving 0 branch lengths in Halliday's tree by setting them to
> ## 1% of the minimum branch length
> zero_brlen <- which(tree_Hall$edge.length == 0)
> min_brlen <- min(tree_Hall$edge.length[-zero_brlen])*0.1
> tree_Hall$edge.length[which(tree_Hall$edge.length == 0)] <- min_brlen
> ## Making sure the tree and the matrix are matching
> clean.MorphNexus <- function(matrix, tree) {
+     cleaned_data <- clean.data(matrix$matrix, tree)
+     tree <- cleaned_data$tree
+     matrix$matrix <- cleaned_data$data
+     return(list(tree, matrix))
+ }
> tmp <- clean.MorphNexus(matrix_Beck, tree_Beck)
> tree_Beck <- tmp[[1]] ; matrix_Beck <- tmp[[2]]
> tmp <- clean.MorphNexus(matrix_Hall, tree_Hall)
> tree_Hall <- tmp[[1]] ; matrix_Hall <- tmp[[2]]
> ## Adding node labels to the tree
> tree_Beck <- makeNodeLabel(tree_Beck, method = "number", prefix = "n")
> tree_Hall <- makeNodeLabel(tree_Hall, method = "number", prefix = "n")
> ## Adding root.time to the tree
> tree_Beck$root.time <- max(tree.age(tree_Beck)[,1])
> tree_Hall$root.time <- max(tree.age(tree_Hall)[,1])
> ## loading the first/last occurrence data
> FADLAD_Beck <- read.csv("../Data/Beck2014_FADLAD.csv", row.names = 1)
> FADLAD_Hall <- read.csv("../Data/Halliday2015_FADLAD.csv", row.names = 1)
\end{Sinput}
\end{Schunk}


\section{Ordination}

\subsection{Ancestral states reconstruction}

First let's recreate the nodal states but excluding any estimation with less than 0.95 scaled likelihood value.

\begin{Schunk}
\begin{Sinput}
> ## Estimating the ancestral states (with a uncertainty threshold of 0.95)
> matrix_nodes_Beck <- AncStateEstMatrix(matrix_Beck, tree_Beck,
+     estimate.allchars = TRUE, uncertainty.threshold = 0.95)
> ## Saving the matrix
> save(matrix_nodes_Beck, file = "../Data/ancestral_states_beck2014.Rda")
> ## Estimating the ancestral states (with a uncertainty threshold of 0.95)
> matrix_nodes_Hall <- AncStateEstMatrix(matrix_Hall, tree_Hall,
+     estimate.allchars = TRUE, uncertainty.threshold = 0.95)
> ## Saving the matrix
> save(matrix_nodes_Hall, file = "../Data/ancestral_states_halliday2015.Rda")
\end{Sinput}
\end{Schunk}

This takes some time (approx. one hour) so it can be skipped.

\begin{Schunk}
\begin{Sinput}
> ## Loading the matrix
> load("../Data/ancestral_states_beck2014.Rda")
> load("../Data/ancestral_states_halliday2015.Rda")
> ## The matrix was assigned to the object matrix_nodes (see above).
> 
> ## Adding the node names as row names
> row.names(matrix_nodes_Beck) <- tree_Beck$node.label
> row.names(matrix_nodes_Hall) <- tree_Hall$node.label
> ## Combining the tips and the nodes in the same matrix
> matrix_Beck$matrix <- rbind(matrix_Beck$matrix, matrix_nodes_Beck) 
> matrix_Hall$matrix <- rbind(matrix_Hall$matrix, matrix_nodes_Hall) 
\end{Sinput}
\end{Schunk}

\subsection{Distance matrix}

Let's then calculate the distance matrix (using Gower's distance).
Same thing, the step takes some time (couple of minutes) so can be skipped for convenience.

\begin{Schunk}
\begin{Sinput}
> ## Calculating the Gower distance
> matrix_dist_Beck <- MorphDistMatrix.fast(matrix_Beck, distance = "Gower")
> save(matrix_dist_Beck, file = "../Data/distmatrix_beck2014.Rda")
> matrix_dist_Hall <- MorphDistMatrix.fast(matrix_Hall, distance = "Gower")
> save(matrix_dist_Hall, file = "../Data/distmatrix_halliday2015.Rda")
\end{Sinput}
\end{Schunk}

\subsection{Ordination}

\begin{Schunk}
\begin{Sinput}
> ## Loading the distance matrices
> load("../Data/distmatrix_beck2014.Rda")
> load("../Data/distmatrix_halliday2015.Rda")
> #### TO FIX PROPERLY!
> #
> # Replacing NAs (incomparable taxa) by median distance (not excellent practice)
> # But there's only 8 NAs in 120409 distances (0.006%) so it won't influence the
> # results (I guess)
> matrix_dist_Hall[which(is.na(matrix_dist_Hall))] <- mean(matrix_dist_Hall, na.rm = TRUE)
> #
> #### TO FIX PROPERLY!
> 
> ## Ordinating the matrices
> matrix_ord_Beck <- cmdscale(matrix_dist_Beck,
+                             k = nrow(matrix_dist_Beck) - 2, add = T)$points
> matrix_ord_Hall <- cmdscale(matrix_dist_Hall,
+                             k = nrow(matrix_dist_Hall) - 2, add = T)$points
\end{Sinput}
\end{Schunk}

\section{Disparity through time}

\subsection{Time slicing}

First let's slice the data into 33 slices (every 5 Mya) using the gradual model.

\begin{Schunk}
\begin{Sinput}
> ## Creating the time slices
> time_slices <- rev(seq(from = 0, to = 120, by = 5))
> ## Creating the time series
> time_series_Beck <- time.series(matrix_ord_Beck, tree_Beck,
+     method = "continuous", time = time_slices, model = "gradual",
+     FADLAD = FADLAD_Beck, verbose = TRUE)
> time_series_Hall <- time.series(matrix_ord_Hall, tree_Hall,
+     method = "continuous", time = time_slices, model = "gradual",
+     FADLAD = FADLAD_Hall, verbose = TRUE)
> 
\end{Sinput}
\end{Schunk}

\subsection{Bootstrapping}

Then let's bootstrap 1000 times.

\begin{Schunk}
\begin{Sinput}
> ## Bootstrapping the time series
> time_series_Beck_bs <- boot.matrix(time_series_Beck, bootstrap = 1000)
> time_series_Hall_bs <- boot.matrix(time_series_Hall, bootstrap = 1000)
\end{Sinput}
\end{Schunk}

\subsection{Calculating disparity}

And now let's calculate disparity.
Couple of changes here, first, disparity is now calculated as two distributions (and then summarised as the median later on) and two aspects are now measured:
\begin{enumerate}
\item the distances between each taxa in a slice and the their centroid (as before)
\item the distances between each taxa in a slice and the overall centre of the morphospace.
\end{enumerate}
The first distribution tells us about the \textit{size} of the taxa's morphospace at each slice (i.e. the space occupied by the taxa at each size) and the second tells us about the \textit{position} of the taxa's morphospace at each slice (i.e. where in the morphospace they are).
The reason for using distributions becomes more obvious later on in the testing part.

\begin{Schunk}
\begin{Sinput}
> ## Calculating the distance from the centroids
> dist_centroids_Beck <- dispRity(time_series_Beck_bs, metric = centroids)
> dist_centroids_Hall <- dispRity(time_series_Hall_bs, metric = centroids)
> ## Calculating the distance from the centre of the space
> dist_centre_Beck <- dispRity(time_series_Beck_bs, metric = centroids,
+                                centroid = rep(0, ncol(matrix_ord_Beck)))
> dist_centre_Hall <- dispRity(time_series_Hall_bs, metric = centroids,
+                                centroid = rep(0, ncol(matrix_ord_Hall)))
> ## Calculating the medians of the distances from centroids/centre
> med_dist_centroids_Beck <- dispRity(dist_centroids_Beck, metric = median)
> med_dist_centroids_Hall <- dispRity(dist_centroids_Hall, metric = median)
> med_dist_centre_Beck <- dispRity(dist_centre_Beck, metric = median)
> med_dist_centre_Hall <- dispRity(dist_centre_Hall, metric = median)
\end{Sinput}
\end{Schunk}

\section{Results}

\subsection{Plot}

\begin{Schunk}
\begin{Sinput}
> ## Graphical parameters
> op <- par(mfrow = c(2,2), bty = "n")
> ## Plotting the distance from centroids
> plot(med_dist_centroids_Beck, main = "Beck 2014",
+     ylab = "Median distance from centroids", xlab = "")
> abline(v = 12, col = "red")
> plot(med_dist_centroids_Hall, main = "Halliday 2015", ylab = "", xlab = "")
> abline(v = 12, col = "red")
> ## Plotting the distances from centre
> plot(med_dist_centre_Beck, ylab = "Median distance from center",
+     xlab = "Age (Mya)")
> abline(v = 12, col = "red")
> plot(med_dist_centre_Hall, ylab = "", xlab = "Age (Mya)")
> abline(v = 12, col = "red")
> ## Resetting graphical parameters
> par(op)
\end{Sinput}
\end{Schunk}

\subsection{Testing}

Before, we compared the distributions of the bootstrapped medians between slices

\begin{Schunk}
\begin{Sinput}
> ## Extracting the bootstrapped median distances from centroids from slices 19
> ## and 20 for the example:
> disparity <- get.dispRity(med_dist_centroids_Beck, what = c(12:13))
> ## Checking the difference between the two slices
> test.dispRity(disparity, test = t.test)
\end{Sinput}
\begin{Soutput}
[[1]]
        statistic
65 - 60  21.72199

[[2]]
        parameter
65 - 60  1938.342

[[3]]
             p.value
65 - 60 8.092521e-94
\end{Soutput}
\end{Schunk}

The problem is that (1) we were only comparing the medians and (2) we were actually testing the effect of the bootstrap not of the changes in disparity (i.e. a significant p-value would mean that there is a difference in the mean of the distribution of medians (which are in turn just the results of the bootstraps)).
This basically results in really big degrees of freedom (see \texttt{parameter}) and will probably tend to always significant results as the amount of bootstraps increase.

To avoid this problem, we can compare the distributions of disparity rather than their central tendencies and bootstrap these comparisons rather than the distributions (i.e. performing the distributions comparisons multiple time with each time a pseudo-replicate (the bootstrap) of the distribution).

\begin{Schunk}
\begin{Sinput}
> ## Extracting the bootstrapped distribution of distances from centroids
> ## from slices 19 and 20 for the example:
> disparity <- get.dispRity(dist_centroids_Beck, what = c(12:13))
> ## Checking the difference between the two slices for each bootstrap
> test.dispRity(disparity, test = t.test, concatenate = FALSE)
\end{Sinput}
\begin{Soutput}
[[1]]
        statistic       2.5%       25%      75%   97.5%
65 - 60 0.9147833 -0.8446137 0.3597284 1.527678 2.45953

[[2]]
        parameter     2.5%     25%      75%    97.5%
65 - 60  54.00714 37.86516 49.3347 60.41235 61.97998

[[3]]
          p.value       2.5%      25%       75%     97.5%
65 - 60 0.3996766 0.01769352 0.132769 0.6371696 0.9730571
\end{Soutput}
\end{Schunk}

Now we have a distribution of statistics and parameters (note that the degrees of freedom are now back to a normal level).
We can thus use this distribution to test our hypothesis, in this case, we have a low chance of rejecting H0 and being wrong (i.e. pvalue $<$ 0.05) in less than 2.5\% of the replicates!
In other words, the distributions are the same (median p-value of 0.4).

And now we can apply that to both the size of the taxa's sub-morphospace through time and their position:

\begin{Schunk}
\begin{Sinput}
> ## Testing any change in group size (centroids) after K-Pg
> test.dispRity(get.dispRity(dist_centroids_Beck, what = c(12:20)),
+     test = t.test, concatenate = FALSE, comparisons = "referential",
+     correction = "bonferroni")
\end{Sinput}
\begin{Soutput}
[[1]]
        statistic       2.5%       25%      75%    97.5%
65 - 60 0.9147833 -0.8446137 0.3597284 1.527678 2.459530
65 - 55 0.6927230 -1.0444124 0.1364635 1.275056 2.374523
65 - 50 1.9207544  0.4474268 1.4215593 2.428646 3.448048
65 - 45 2.2850633  0.9360157 1.7835723 2.754341 3.810877
65 - 40 1.9984564  0.6331092 1.4987886 2.471508 3.491970
65 - 35 1.8931845  0.6360534 1.4632072 2.289641 3.322081
65 - 30 1.9709847  0.6600025 1.5450305 2.318031 3.450900
65 - 25 1.7147330  0.5938075 1.3463149 2.049747 3.071067

[[2]]
        parameter     2.5%      25%      75%    97.5%
65 - 60  54.00714 37.86516 49.33470 60.41235 61.97998
65 - 55  54.35329 37.44442 49.83175 60.55225 61.98337
65 - 50  33.89302 22.95480 27.81262 38.75755 51.90786
65 - 45  25.98327 18.73058 21.49606 28.88411 41.23487
65 - 40  28.21174 19.79189 23.26336 31.53461 45.51507
65 - 35  19.03124 14.73785 16.22375 20.29659 31.74108
65 - 30  15.41861 12.41041 13.32783 16.01850 24.26067
65 - 25  15.43249 12.41602 13.38847 16.31297 23.84081

[[3]]
           p.value         2.5%        25%        75%     97.5%
65 - 60 0.39967660 0.0176935194 0.13276905 0.63716964 0.9730571
65 - 55 0.45868024 0.0213227137 0.20230115 0.70961560 0.9709313
65 - 50 0.13107232 0.0016567486 0.02208533 0.16346503 0.6517692
65 - 45 0.07134567 0.0007229066 0.01095332 0.08633388 0.3560630
65 - 40 0.11386611 0.0016574877 0.02044169 0.14583644 0.5304523
65 - 35 0.12505011 0.0038257234 0.03515379 0.16067930 0.5329757
65 - 30 0.11511303 0.0034481577 0.03542677 0.14135072 0.5194434
65 - 25 0.15560467 0.0080558186 0.05927448 0.19754694 0.5592213
\end{Soutput}
\begin{Sinput}
> test.dispRity(get.dispRity(dist_centroids_Hall, what = c(12:20)),
+     test = t.test, concatenate = FALSE, comparisons = "referential",
+     correction = "bonferroni")
\end{Sinput}
\begin{Soutput}
[[1]]
         statistic        2.5%        25%        75%     97.5%
65 - 60 -3.3330013 -4.95668359 -3.8557254 -2.8007521 -1.813268
65 - 55 -5.3095569 -7.08508889 -5.8398166 -4.7097518 -3.681545
65 - 50 -3.7295264 -5.45722035 -4.2927029 -3.1552635 -2.054318
65 - 45 -0.5359654 -2.37221967 -1.1297172  0.0928263  1.157233
65 - 40  1.7304596  0.35933006  1.3002974  2.1564520  3.014847
65 - 35  1.5696784  0.21839821  1.1128556  2.0285433  2.962330
65 - 30  1.2049259 -0.15837070  0.8019610  1.6086395  2.394283
65 - 25  1.3178384 -0.00503767  0.9411156  1.7227408  2.506521

[[2]]
        parameter     2.5%      25%      75%     97.5%
65 - 60  76.75435 55.28250 66.26678 85.52933 107.07055
65 - 55  72.52552 53.19196 63.16238 81.44204  98.49232
65 - 50  82.52234 61.40796 75.43577 91.59313  94.96953
65 - 45  61.18663 42.00048 53.61578 70.32014  74.95919
65 - 40  31.41704 23.19798 26.30260 34.52790  49.04920
65 - 35  29.14638 21.55024 24.45947 32.11300  46.99679
65 - 30  20.39273 16.75170 17.99790 21.57553  30.16988
65 - 25  18.69399 15.50308 16.53139 19.61086  27.08259

[[3]]
             p.value         2.5%          25%          75%        97.5%
65 - 60 9.783863e-03 4.809926e-06 2.495973e-04 6.503933e-03 0.0729389550
65 - 55 4.360725e-05 7.036060e-10 1.464759e-07 1.170213e-05 0.0004131995
65 - 50 5.040219e-03 5.564254e-07 4.961521e-05 2.246752e-03 0.0426982078
65 - 45 4.908678e-01 2.024863e-02 2.395055e-01 7.494322e-01 0.9765378626
65 - 40 1.542917e-01 5.088938e-03 3.967839e-02 2.032722e-01 0.7097305433
65 - 35 1.993787e-01 6.008576e-03 5.384837e-02 2.729358e-01 0.8105538735
65 - 30 3.000629e-01 2.768617e-02 1.242626e-01 4.279613e-01 0.8756290881
65 - 25 2.638469e-01 2.230895e-02 1.029528e-01 3.581971e-01 0.8755634657
\end{Soutput}
\end{Schunk}

When looking at Beck's data set, there is (as before) no significant expansion of the eutherian morphospace after the K-Pg event.
However, when looking at Halliday's data, there is a consistent significant expansion in the 5 Mya after the K-Pg event (97.5\% of the p-values $<$ 0.001!) as well with a less consistent one in the following 10 Mya (75\% of the p-values $<$ 0.005).
After that though, the morphospace size seems to go down to pre K-Pg levels (but that might be due to the under sampling).

\begin{Schunk}
\begin{Sinput}
> ## Testing any change in group position (centre) after K-Pg
> test.dispRity(get.dispRity(dist_centre_Beck, what = c(12:20)),
+     test = t.test, concatenate = FALSE, comparisons = "referential",
+     correction = "bonferroni")
\end{Sinput}
\begin{Soutput}
[[1]]
         statistic      2.5%        25%         75%     97.5%
65 - 60  0.2630636 -1.630182 -0.3630964  0.92406003  2.180075
65 - 55 -0.6031800 -2.617759 -1.2362542  0.01579344  1.241088
65 - 50  0.3661335 -1.744498 -0.2735814  1.01416486  2.342727
65 - 45  0.2071830 -1.884423 -0.4948817  0.86358348  2.467839
65 - 40 -0.1013658 -2.147666 -0.7679460  0.57452064  1.985088
65 - 35 -0.8164971 -2.890869 -1.4846424 -0.12357517  1.121006
65 - 30 -1.4087687 -4.019230 -2.0132660 -0.61523885  0.809619
65 - 25 -3.2175043 -7.336468 -3.9778235 -2.11007262 -0.902265

[[2]]
        parameter     2.5%      25%      75%    97.5%
65 - 60  55.89910 47.91654 53.54571 58.56496 61.67088
65 - 55  55.15963 47.41650 52.43175 58.11236 61.64759
65 - 50  35.82849 28.77616 32.25035 38.70533 46.56354
65 - 45  26.90363 22.27642 24.57271 28.60670 34.71297
65 - 40  28.02131 23.50971 26.01143 29.51412 34.61025
65 - 35  19.69390 17.06158 18.33055 20.55524 24.30539
65 - 30  16.58681 14.26666 15.21818 17.20197 21.25532
65 - 25  24.32197 16.50793 19.16660 26.48500 45.94039

[[3]]
           p.value         2.5%          25%        75%     97.5%
65 - 60 0.49979868 2.455271e-02 0.2462790962 0.75190118 0.9715318
65 - 55 0.46352044 1.136438e-02 0.1966536145 0.73735498 0.9746299
65 - 50 0.48426922 2.302380e-02 0.2299061704 0.74264702 0.9696690
65 - 45 0.49611730 1.579224e-02 0.2319113427 0.75733538 0.9650609
65 - 40 0.49906492 1.972234e-02 0.2486674283 0.75085223 0.9742211
65 - 35 0.42847419 8.337426e-03 0.1471273115 0.70516576 0.9666992
65 - 30 0.30006524 6.313256e-04 0.0604691204 0.49990922 0.9467349
65 - 25 0.04981126 3.536786e-09 0.0004896603 0.04795985 0.3797201
\end{Soutput}
\begin{Sinput}
> test.dispRity(get.dispRity(dist_centre_Hall, what = c(12:20)),
+     test = t.test, concatenate = FALSE, comparisons = "referential",
+     correction = "bonferroni")
\end{Sinput}
\begin{Soutput}
[[1]]
         statistic       2.5%       25%         75%      97.5%
65 - 60 -1.7848510  -3.811931 -2.483189 -1.07416531  0.2040437
65 - 55 -5.3337467  -7.653155 -6.033123 -4.57676694 -3.1647371
65 - 50 -4.3632112  -6.537343 -5.072229 -3.63500668 -2.2313905
65 - 45 -2.2605134  -4.583802 -2.967788 -1.52435768 -0.1320878
65 - 40 -0.5400611  -2.593491 -1.137717  0.07609257  1.2863416
65 - 35 -1.3753814  -3.597463 -2.077025 -0.58795281  0.6184512
65 - 30 -4.3158272 -11.360597 -5.112202 -2.85676897 -1.4813432
65 - 25 -4.9596005 -12.347116 -5.853889 -3.19195204 -1.5723883

[[2]]
        parameter     2.5%       25%       75%     97.5%
65 - 60 102.42591 95.80486 100.40302 104.65116 107.99735
65 - 55  99.80440 87.93130  96.44173 103.76617 108.24881
65 - 50  94.01728 89.70749  93.78142  94.89381  94.99912
65 - 45  57.91308 50.82924  54.83335  60.41054  68.35377
65 - 40  34.48408 29.88318  32.23910  36.19537  42.08894
65 - 35  30.13894 26.81224  28.48830  31.33716  36.18227
65 - 30  23.51127 19.00460  20.12658  23.62406  54.22931
65 - 25  24.56468 17.71964  18.93924  24.14916  56.73724

[[3]]
             p.value         2.5%          25%          75%       97.5%
65 - 60 0.1913394118 2.316288e-04 1.465853e-02 2.722038e-01 0.890756741
65 - 55 0.0003459418 1.447246e-11 3.029931e-08 1.343625e-05 0.002020223
65 - 50 0.0025392598 3.828977e-09 1.977453e-06 4.512738e-04 0.028079504
65 - 45 0.1192246671 2.382932e-05 4.323867e-03 1.330749e-01 0.730656346
65 - 40 0.4793838604 1.314703e-02 2.236806e-01 7.299661e-01 0.980439220
65 - 35 0.2954320225 1.091120e-03 4.623694e-02 5.048598e-01 0.939203917
65 - 30 0.0175643003 9.106813e-16 3.471176e-05 9.827791e-03 0.154408048
65 - 25 0.0131050168 1.999000e-17 4.661125e-06 4.638184e-03 0.132802065
\end{Soutput}
\end{Schunk}

When looking at the shifts in morphospace, in Beck's dataset, there is no significant ones right after the K-Pg event.
However, it seems that the morphospace shifts after the PETM (around 35 Mya, see figure)

For Halliday's one however, there is a shift that starts 5 Mya after the K-Pg event and holds for 10 Mya before coming back to the pre K-Pg position.


\section{rarefaction}

Though as Beck is undersampling, maybe Halliday is oversampling.
Here are the rarefied results with 10 taxa in each bootstrap pseudo-replication.

\begin{Schunk}
\begin{Sinput}
> ## Rarefying the time series
> time_series_Beck_rar <- boot.matrix(time_series_Beck, bootstrap = 1000,
+     rarefaction = 10)
> time_series_Hall_rar <- boot.matrix(time_series_Hall, bootstrap = 1000,
+     rarefaction = 10)
\end{Sinput}
\end{Schunk}


\begin{Schunk}
\begin{Sinput}
> ## Calculating the distance from the centroids
> dist_centroids_Beck_rar <- dispRity(time_series_Beck_rar, metric = centroids)
> dist_centroids_Hall_rar <- dispRity(time_series_Hall_rar, metric = centroids)
> ## Calculating the distance from the centre of the space
> dist_centre_Beck_rar <- dispRity(time_series_Beck_rar, metric = centroids,
+                                centroid = rep(0, ncol(matrix_ord_Beck)))
> dist_centre_Hall_rar <- dispRity(time_series_Hall_rar, metric = centroids,
+                                centroid = rep(0, ncol(matrix_ord_Hall)))
> ## Calculating the medians of the distances from centroids/centre
> med_dist_centroids_Beck_rar <- dispRity(dist_centroids_Beck_rar, metric = median)
> med_dist_centroids_Hall_rar <- dispRity(dist_centroids_Hall_rar, metric = median)
> med_dist_centre_Beck_rar <- dispRity(dist_centre_Beck_rar, metric = median)
> med_dist_centre_Hall_rar <- dispRity(dist_centre_Hall_rar, metric = median)
\end{Sinput}
\end{Schunk}

\begin{Schunk}
\begin{Sinput}
> ## Graphical parameters
> op <- par(mfrow = c(2,2), bty = "n")
> ## Plotting the distance from centroids
> plot(med_dist_centroids_Beck_rar, main = "Beck 2014 (rarefied)",
+     ylab = "Median distance from centroids", xlab = "")
> abline(v = 12, col = "red")
> plot(med_dist_centroids_Hall_rar, main = "Halliday 2015 (rarefied)", ylab = "",
+     xlab = "")
> abline(v = 12, col = "red")
> ## Plotting the distances from centre
> plot(med_dist_centre_Beck_rar, ylab = "Median distance from center",
+     xlab = "Age (Mya)")
> abline(v = 12, col = "red")
> plot(med_dist_centre_Hall_rar, ylab = "", xlab = "Age (Mya)")
> abline(v = 12, col = "red")
> ## Resetting graphical parameters
> par(op)
\end{Sinput}
\end{Schunk}

\begin{Schunk}
\begin{Sinput}
> ## Testing any change in group size (centroids) after K-Pg
> test.dispRity(get.dispRity(dist_centroids_Beck, what = c(12:20)),
+     test = t.test, concatenate = FALSE, comparisons = "referential",
+     correction = "bonferroni")
\end{Sinput}
\begin{Soutput}
[[1]]
        statistic       2.5%       25%      75%    97.5%
65 - 60 0.9147833 -0.8446137 0.3597284 1.527678 2.459530
65 - 55 0.6927230 -1.0444124 0.1364635 1.275056 2.374523
65 - 50 1.9207544  0.4474268 1.4215593 2.428646 3.448048
65 - 45 2.2850633  0.9360157 1.7835723 2.754341 3.810877
65 - 40 1.9984564  0.6331092 1.4987886 2.471508 3.491970
65 - 35 1.8931845  0.6360534 1.4632072 2.289641 3.322081
65 - 30 1.9709847  0.6600025 1.5450305 2.318031 3.450900
65 - 25 1.7147330  0.5938075 1.3463149 2.049747 3.071067

[[2]]
        parameter     2.5%      25%      75%    97.5%
65 - 60  54.00714 37.86516 49.33470 60.41235 61.97998
65 - 55  54.35329 37.44442 49.83175 60.55225 61.98337
65 - 50  33.89302 22.95480 27.81262 38.75755 51.90786
65 - 45  25.98327 18.73058 21.49606 28.88411 41.23487
65 - 40  28.21174 19.79189 23.26336 31.53461 45.51507
65 - 35  19.03124 14.73785 16.22375 20.29659 31.74108
65 - 30  15.41861 12.41041 13.32783 16.01850 24.26067
65 - 25  15.43249 12.41602 13.38847 16.31297 23.84081

[[3]]
           p.value         2.5%        25%        75%     97.5%
65 - 60 0.39967660 0.0176935194 0.13276905 0.63716964 0.9730571
65 - 55 0.45868024 0.0213227137 0.20230115 0.70961560 0.9709313
65 - 50 0.13107232 0.0016567486 0.02208533 0.16346503 0.6517692
65 - 45 0.07134567 0.0007229066 0.01095332 0.08633388 0.3560630
65 - 40 0.11386611 0.0016574877 0.02044169 0.14583644 0.5304523
65 - 35 0.12505011 0.0038257234 0.03515379 0.16067930 0.5329757
65 - 30 0.11511303 0.0034481577 0.03542677 0.14135072 0.5194434
65 - 25 0.15560467 0.0080558186 0.05927448 0.19754694 0.5592213
\end{Soutput}
\begin{Sinput}
> test.dispRity(get.dispRity(dist_centroids_Hall, what = c(12:20)),
+     test = t.test, concatenate = FALSE, comparisons = "referential",
+     correction = "bonferroni")
\end{Sinput}
\begin{Soutput}
[[1]]
         statistic        2.5%        25%        75%     97.5%
65 - 60 -3.3330013 -4.95668359 -3.8557254 -2.8007521 -1.813268
65 - 55 -5.3095569 -7.08508889 -5.8398166 -4.7097518 -3.681545
65 - 50 -3.7295264 -5.45722035 -4.2927029 -3.1552635 -2.054318
65 - 45 -0.5359654 -2.37221967 -1.1297172  0.0928263  1.157233
65 - 40  1.7304596  0.35933006  1.3002974  2.1564520  3.014847
65 - 35  1.5696784  0.21839821  1.1128556  2.0285433  2.962330
65 - 30  1.2049259 -0.15837070  0.8019610  1.6086395  2.394283
65 - 25  1.3178384 -0.00503767  0.9411156  1.7227408  2.506521

[[2]]
        parameter     2.5%      25%      75%     97.5%
65 - 60  76.75435 55.28250 66.26678 85.52933 107.07055
65 - 55  72.52552 53.19196 63.16238 81.44204  98.49232
65 - 50  82.52234 61.40796 75.43577 91.59313  94.96953
65 - 45  61.18663 42.00048 53.61578 70.32014  74.95919
65 - 40  31.41704 23.19798 26.30260 34.52790  49.04920
65 - 35  29.14638 21.55024 24.45947 32.11300  46.99679
65 - 30  20.39273 16.75170 17.99790 21.57553  30.16988
65 - 25  18.69399 15.50308 16.53139 19.61086  27.08259

[[3]]
             p.value         2.5%          25%          75%        97.5%
65 - 60 9.783863e-03 4.809926e-06 2.495973e-04 6.503933e-03 0.0729389550
65 - 55 4.360725e-05 7.036060e-10 1.464759e-07 1.170213e-05 0.0004131995
65 - 50 5.040219e-03 5.564254e-07 4.961521e-05 2.246752e-03 0.0426982078
65 - 45 4.908678e-01 2.024863e-02 2.395055e-01 7.494322e-01 0.9765378626
65 - 40 1.542917e-01 5.088938e-03 3.967839e-02 2.032722e-01 0.7097305433
65 - 35 1.993787e-01 6.008576e-03 5.384837e-02 2.729358e-01 0.8105538735
65 - 30 3.000629e-01 2.768617e-02 1.242626e-01 4.279613e-01 0.8756290881
65 - 25 2.638469e-01 2.230895e-02 1.029528e-01 3.581971e-01 0.8755634657
\end{Soutput}
\end{Schunk}

Now there is no more significant change morphospace size in both dataset.

\begin{Schunk}
\begin{Sinput}
> ## Testing any change in group position (centre) after K-Pg
> test.dispRity(get.dispRity(dist_centre_Beck, what = c(12:20)),
+     test = t.test, concatenate = FALSE, comparisons = "referential",
+     correction = "bonferroni")
\end{Sinput}
\begin{Soutput}
[[1]]
         statistic      2.5%        25%         75%     97.5%
65 - 60  0.2630636 -1.630182 -0.3630964  0.92406003  2.180075
65 - 55 -0.6031800 -2.617759 -1.2362542  0.01579344  1.241088
65 - 50  0.3661335 -1.744498 -0.2735814  1.01416486  2.342727
65 - 45  0.2071830 -1.884423 -0.4948817  0.86358348  2.467839
65 - 40 -0.1013658 -2.147666 -0.7679460  0.57452064  1.985088
65 - 35 -0.8164971 -2.890869 -1.4846424 -0.12357517  1.121006
65 - 30 -1.4087687 -4.019230 -2.0132660 -0.61523885  0.809619
65 - 25 -3.2175043 -7.336468 -3.9778235 -2.11007262 -0.902265

[[2]]
        parameter     2.5%      25%      75%    97.5%
65 - 60  55.89910 47.91654 53.54571 58.56496 61.67088
65 - 55  55.15963 47.41650 52.43175 58.11236 61.64759
65 - 50  35.82849 28.77616 32.25035 38.70533 46.56354
65 - 45  26.90363 22.27642 24.57271 28.60670 34.71297
65 - 40  28.02131 23.50971 26.01143 29.51412 34.61025
65 - 35  19.69390 17.06158 18.33055 20.55524 24.30539
65 - 30  16.58681 14.26666 15.21818 17.20197 21.25532
65 - 25  24.32197 16.50793 19.16660 26.48500 45.94039

[[3]]
           p.value         2.5%          25%        75%     97.5%
65 - 60 0.49979868 2.455271e-02 0.2462790962 0.75190118 0.9715318
65 - 55 0.46352044 1.136438e-02 0.1966536145 0.73735498 0.9746299
65 - 50 0.48426922 2.302380e-02 0.2299061704 0.74264702 0.9696690
65 - 45 0.49611730 1.579224e-02 0.2319113427 0.75733538 0.9650609
65 - 40 0.49906492 1.972234e-02 0.2486674283 0.75085223 0.9742211
65 - 35 0.42847419 8.337426e-03 0.1471273115 0.70516576 0.9666992
65 - 30 0.30006524 6.313256e-04 0.0604691204 0.49990922 0.9467349
65 - 25 0.04981126 3.536786e-09 0.0004896603 0.04795985 0.3797201
\end{Soutput}
\begin{Sinput}
> test.dispRity(get.dispRity(dist_centre_Hall, what = c(12:20)),
+     test = t.test, concatenate = FALSE, comparisons = "referential",
+     correction = "bonferroni")
\end{Sinput}
\begin{Soutput}
[[1]]
         statistic       2.5%       25%         75%      97.5%
65 - 60 -1.7848510  -3.811931 -2.483189 -1.07416531  0.2040437
65 - 55 -5.3337467  -7.653155 -6.033123 -4.57676694 -3.1647371
65 - 50 -4.3632112  -6.537343 -5.072229 -3.63500668 -2.2313905
65 - 45 -2.2605134  -4.583802 -2.967788 -1.52435768 -0.1320878
65 - 40 -0.5400611  -2.593491 -1.137717  0.07609257  1.2863416
65 - 35 -1.3753814  -3.597463 -2.077025 -0.58795281  0.6184512
65 - 30 -4.3158272 -11.360597 -5.112202 -2.85676897 -1.4813432
65 - 25 -4.9596005 -12.347116 -5.853889 -3.19195204 -1.5723883

[[2]]
        parameter     2.5%       25%       75%     97.5%
65 - 60 102.42591 95.80486 100.40302 104.65116 107.99735
65 - 55  99.80440 87.93130  96.44173 103.76617 108.24881
65 - 50  94.01728 89.70749  93.78142  94.89381  94.99912
65 - 45  57.91308 50.82924  54.83335  60.41054  68.35377
65 - 40  34.48408 29.88318  32.23910  36.19537  42.08894
65 - 35  30.13894 26.81224  28.48830  31.33716  36.18227
65 - 30  23.51127 19.00460  20.12658  23.62406  54.22931
65 - 25  24.56468 17.71964  18.93924  24.14916  56.73724

[[3]]
             p.value         2.5%          25%          75%       97.5%
65 - 60 0.1913394118 2.316288e-04 1.465853e-02 2.722038e-01 0.890756741
65 - 55 0.0003459418 1.447246e-11 3.029931e-08 1.343625e-05 0.002020223
65 - 50 0.0025392598 3.828977e-09 1.977453e-06 4.512738e-04 0.028079504
65 - 45 0.1192246671 2.382932e-05 4.323867e-03 1.330749e-01 0.730656346
65 - 40 0.4793838604 1.314703e-02 2.236806e-01 7.299661e-01 0.980439220
65 - 35 0.2954320225 1.091120e-03 4.623694e-02 5.048598e-01 0.939203917
65 - 30 0.0175643003 9.106813e-16 3.471176e-05 9.827791e-03 0.154408048
65 - 25 0.0131050168 1.999000e-17 4.661125e-06 4.638184e-03 0.132802065
\end{Soutput}
\end{Schunk}

Nor is there any change in morphospace position...

\end{document}


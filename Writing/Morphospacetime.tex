\documentclass[a4paper,11pt]{article}


\usepackage{natbib}
\usepackage{enumerate}
\usepackage[osf]{mathpazo}
\usepackage{lastpage} 
\pagenumbering{arabic}
\linespread{1.66}

\begin{document}

\begin{flushright}
Version dated: \today
\end{flushright}
\begin{center}

%Title
\noindent{\Large{\bf{Spatio-temporal Disparity (STD)}}}\\
\bigskip
%Author
\noindent{Thomas Guillerme - guillert@tcd.ie - http://tguillerme.github.io/}\\

\end{center}

\section{Data}
Use a morphological matrix with cladistic characters for each fossil and living taxa.
Some caveats has to be taken into account, namely: 
\begin{enumerate}
\item{Cladistic disparity is based on phylogeny}
Therefore, analysing a cladistic dataset is likely to reflect the phylogenetic morphospace instead of the the phenetic morphospace \citep{Foote29111996,Wagner01011997}.
However, discrete cladistic characters are still the best source for quantifying overall morphology for large and diverse groups \citep{Brusatte12092008}.
\end{enumerate}

\section{Time intervals}
The intervals should be a compromise between the resolution and the sample size and must be "sufficiently coards that nearly all generic first and last occurenaces can be unambiguously assigned" \citep{Foote01071994}.

\section{PCO}
The cladistic matrix is derived into an Euclidean distance matrix using Calliez negative Eigenvalues correction \citep{toljagictriassic-jurassic2013}.
Then a PCO is performed on the new matrix.
Principal coordinates (PCO) is preferred to principal components (PCA) because it can "better handle missing data and inapplicable characters" \citep{lofgren2003,Wesley-Hunt2005}.

\section{Disparity metrics}
Disparity metrics should be calculated using a small number of PCO axes (12 \citep{Brusatte12092008}) encompassing a majority of the cumulative variance (63\% \citep{Brusatte12092008})
Four disparity metrics can be calculated, the sum and product of the ranges and variance on the selected axes \citep{Wills1994}.
The product of ranges and variance is a better proxy for the volume of morphospace (CHECK THAT in Morphometric book).
The sum of ranges and variance is a better proxy for the differences in morphospace (CHECK THAT in Morphometric book).

The variance measures are representing the average dissimilarity among forms and are more sensitive to taxonomic practice but insensitive to sample size \citep{BIJ:BIJ45}.
The ranges measures are representing the entire spread of the morphological variation (i.e. the morphospace?) but is more sensitive to sampling biases \citep{BIJ:BIJ455}.
Both gives an indication of the volume of the morphospace occupied and can be calculated using the Rare software \citep{BIJ:BIJ455}.

Taxa are grouped per phylogeny and time intervals.
The difference in distribution of groups in the morphospace was tested using non parametric multivariate analysis of varaiance (non-parametric MANOVA).
"NPMANOVA evaluates similar distributions of variances in two or more groups of multivariate data (in our case, scores in the distance matrix) and operates through permutations of groups’ elements (taxa) \citep{AEC:AEC1070}."\citep{Brusatte12092008}.
\textit{Bonferroni corrections were applied for multiple comparisons} (???).

The Rare software \citep{BIJ:BIJ455} can be used for additional informations
\begin{enumerate}
\item{overlap of the 95\% BS CI}
"Which are calculated by Rare (1000 replications). This is a conservative test—it treats the data as two one-sample problems instead of a single two-sample problem—which we prefer, as it gives stronger confidence to a significant result." \citep{Brusatte12092008}
\item{Rarefaction curves}
"Which are also calculated by Rare, give an indication of sample-size biases, which are especially important to consider for range-based metrics."\citep{Brusatte12092008}
\end{enumerate}



\section{Update 1}
\begin{enumerate}
\item{Bayesian ancestral state reconstruction}
\item{General morphospace = cladistic space (axis do not represent a "real" gradient in morphology)}
\item{Character family morphospace = character morphospace (axis represent a "real" gradient in morphology)}
\end{enumerate}


\section{Update 2}
About the construction of the PCO axis:

-As outlined in Brusatte et al. (2008): The characters in this analysis have been derived from cladistic datasets, which may bias the disparity analyses towards recovering groups based on phylogeny rather than overall phenetic dissimilarity. However, this problem has been addressed by previous authors (e.g., Foote, 1994, 1995, 1996; Wagner, 1997), and the general consensus is that discrete cladistic characters are the best source of quantifying overall form for large and diverse groups such as archosaurs (Brusatte et al 2008 Biol Lett SOM).

-PCO combines information from the character matrix (437 characters) into a smaller and more manageable number of variables (63 axes), with the first axis representing those characters that contribute most to overall disparity, and each additional axis representing characters of successively less significance. (Brusatte et al 2008 Science SOM).

How do you go from 437 characters (437 variables leading to 437 PC axes?) to 63 axes? They are the number of species. PCO is a square matrix number of species 

\section{Upadet 3 - Post SVP/RadExLinSoc}
\begin{enumerate}
\item{Disparity = morphological diversity (number of body plans)} \citep{friedmanexplosive2010}

It can be calculated using multivariate statistics test (NPMANOVA with Bonferroni correction) \citep{thorneresetting2011}. A second way to calculate it is to calculate this distance from centroid for each group (including the metagroup containing all the data). It can be scaled on the overall character-space as the relative disparity (ReDi) so that a relative disparity of 1 shows the maximum disparity observable and a relative disparity of 0, the minimal one (= no disparity).
\item{Diversity = species diversity (number of species)}

It can be calculated using three measures: (1) the number of observed taxa in the time bin/slice; (2) the number of observed taxa + the lazarus taxa ; and (3) the number of observed and lazarus taxa + the ghost lineages \citep{thorneresetting2011}.
\end{enumerate}
\subsection{Radiation hypothesis test}
Test if there is a significant difference in disparity and diversity between two time stages.
\subsection{Sampling effect}
Apply a Jackknife procedure to the data with re-sampling with replacement on the data set (from the start of the analysis, redo the ACE and the PCO).
\subsection{Div. rate shift in the molecular data}
Test if their is a diversification rate shift in the molecular data around KT using the Silvestro 2011 method \citep{silvestroa2011}.

\section{Discussion elements}
\begin{enumerate}
\item{Acantomorphs radiations around after KT \citep{friedmanexplosive2010}}
\end{enumerate}




\bibliographystyle{sysbio} %don't write the suffix
\bibliography{References}

\end{document}

\documentclass[a4paper,11pt]{article}\usepackage[]{graphicx}\usepackage[]{color}
%% maxwidth is the original width if it is less than linewidth
%% otherwise use linewidth (to make sure the graphics do not exceed the margin)
\makeatletter
\def\maxwidth{ %
  \ifdim\Gin@nat@width>\linewidth
    \linewidth
  \else
    \Gin@nat@width
  \fi
}
\makeatother

\definecolor{fgcolor}{rgb}{0.345, 0.345, 0.345}
\newcommand{\hlnum}[1]{\textcolor[rgb]{0.686,0.059,0.569}{#1}}%
\newcommand{\hlstr}[1]{\textcolor[rgb]{0.192,0.494,0.8}{#1}}%
\newcommand{\hlcom}[1]{\textcolor[rgb]{0.678,0.584,0.686}{\textit{#1}}}%
\newcommand{\hlopt}[1]{\textcolor[rgb]{0,0,0}{#1}}%
\newcommand{\hlstd}[1]{\textcolor[rgb]{0.345,0.345,0.345}{#1}}%
\newcommand{\hlkwa}[1]{\textcolor[rgb]{0.161,0.373,0.58}{\textbf{#1}}}%
\newcommand{\hlkwb}[1]{\textcolor[rgb]{0.69,0.353,0.396}{#1}}%
\newcommand{\hlkwc}[1]{\textcolor[rgb]{0.333,0.667,0.333}{#1}}%
\newcommand{\hlkwd}[1]{\textcolor[rgb]{0.737,0.353,0.396}{\textbf{#1}}}%

\usepackage{framed}
\makeatletter
\newenvironment{kframe}{%
 \def\at@end@of@kframe{}%
 \ifinner\ifhmode%
  \def\at@end@of@kframe{\end{minipage}}%
  \begin{minipage}{\columnwidth}%
 \fi\fi%
 \def\FrameCommand##1{\hskip\@totalleftmargin \hskip-\fboxsep
 \colorbox{shadecolor}{##1}\hskip-\fboxsep
     % There is no \\@totalrightmargin, so:
     \hskip-\linewidth \hskip-\@totalleftmargin \hskip\columnwidth}%
 \MakeFramed {\advance\hsize-\width
   \@totalleftmargin\z@ \linewidth\hsize
   \@setminipage}}%
 {\par\unskip\endMakeFramed%
 \at@end@of@kframe}
\makeatother

\definecolor{shadecolor}{rgb}{.97, .97, .97}
\definecolor{messagecolor}{rgb}{0, 0, 0}
\definecolor{warningcolor}{rgb}{1, 0, 1}
\definecolor{errorcolor}{rgb}{1, 0, 0}
\newenvironment{knitrout}{}{} % an empty environment to be redefined in TeX

\usepackage{alltt}
\usepackage[sc]{mathpazo}
\usepackage[T1]{fontenc}
\usepackage{geometry}
\usepackage{url}
\usepackage{natbib}
\usepackage{enumerate}
\usepackage{lastpage} 
\pagenumbering{arabic}
\linespread{1.66}

\usepackage[unicode=true,pdfusetitle,
bookmarks=true,bookmarksnumbered=true,bookmarksopen=true,bookmarksopenlevel=2,
breaklinks=false,pdfborder={0 0 1},backref=false,colorlinks=false]
{hyperref}

\geometry{verbose,tmargin=2.5cm,bmargin=2.5cm,lmargin=2.5cm,rmargin=2.5cm}
\setcounter{secnumdepth}{2}
\setcounter{tocdepth}{2}

\hypersetup{pdfstartview={XYZ null null 1}}
\IfFileExists{upquote.sty}{\usepackage{upquote}}{}
\begin{document}



\begin{flushright}
Version dated: \today
\end{flushright}
\begin{center}

%Title
\noindent{\Large{\bf{Spatio-temporal Disparity (STD)}}}\\
\bigskip
%Author
\noindent{Thomas Guillerme - guillert@tcd.ie - http://tguillerme.github.io/}\\

\end{center}
This is a quick demo on how I run my Spatio-Temporal Disparity analysis.

\textbf{DISCLAIMER}:
This demo contains really early results of the work and is far from being correctly framed, both in the analysis than in the ideas.
The functions and the associated testing have only be tested on two specific data sets and might not work with other ones.
Also, most analyses violate any statistical assumptions and the functions options are usually just set to default without any thinking about them.
Again, this a a \textbf{really early version} of this work.

This document contains a pipelined analysis to produce a plot of morphospace through time using a conglomerate of functions that are not yet grouped in a package.
Before starting the analysis, make sure you cloned the the Spatio-Temporal Disparity folder on your machine (available here: \url{https://github.com/TGuillerme/SpatioTemporal_Disparity}).

\section{Before the analysis}
Then run the following to check the repository and load all the necessary functions or data.

\begin{knitrout}
\definecolor{shadecolor}{rgb}{0.969, 0.969, 0.969}\color{fgcolor}\begin{kframe}
\begin{alltt}
\hlkwd{setwd}\hlstd{(}\hlstr{"Analysis"}\hlstd{)}
\hlkwd{source}\hlstd{(}\hlstr{"functions.R"}\hlstd{)}
\end{alltt}
\end{kframe}
\end{knitrout}

\section{The data} \label{data}

%\texttt{x} for putting the text in courier
%x[1] for printing out of the coding boxes an R object
You now can choose which data to load by using \texttt{set.data}. Two data sets are available:
\begin{enumerate}
\item{\texttt{"Beck"}}

A data set from Beck and Lee 2014 Proceedings of the Royal Society B containing 102 taxa and 421 morphological characters.

\item{\texttt{"Slater"}}

A data set from Slater 2013 Methods in Ecology and Evolution containing 90 taxa and 446 morphological characters
\end{enumerate}
You can use the \texttt{with.anc.matrix} option to also load the associated ancestral states estimation matrix (see Ancestral states estimation ~\ref{ACE})).
For this example I will load the \texttt{Beck} dataset. Note that you can you any of the two example and still run the example

\begin{knitrout}
\definecolor{shadecolor}{rgb}{0.969, 0.969, 0.969}\color{fgcolor}\begin{kframe}
\begin{alltt}
\hlkwd{source}\hlstd{(}\hlstr{"Beck.data.R"}\hlstd{)}
\end{alltt}
\end{kframe}
\end{knitrout}

Note that \texttt{set.data} loaded and generate various objects namely \texttt{tree} that can be plotted by just calling:

\begin{knitrout}
\definecolor{shadecolor}{rgb}{0.969, 0.969, 0.969}\color{fgcolor}\begin{kframe}
\begin{alltt}
\hlkwd{plot}\hlstd{(tree,} \hlkwc{cex}\hlstd{=}\hlnum{0.4}\hlstd{,} \hlkwc{main}\hlstd{=}\hlstr{"Beck and Lee 2014 - Placental"}\hlstd{)}\hlcom{#tiny tiny labels!}
\hlkwd{axisPhylo}\hlstd{()}
\end{alltt}
\end{kframe}

{\centering \includegraphics[width=\maxwidth]{figure/minimal-BeckTree-1} 

}



\end{knitrout}

\section{Generating the PCO data}
Once we've loaded the data we can run a PCO (or MDS) analysis to see the occupancy of placental mammals character-space through time.

\subsection{Ancestral state estimation} \label{ACE}
First we need to estimate the complete character-space by estimating the ancestral states of each characters on each node of the tree.
We can use the \texttt{anc.state} function that needs a \texttt{tree} and a \texttt{table} containing the ancestral states.
If our case both where loaded when using the \texttt{set.data} function (~\ref{data}).
Three different methods can be used for \texttt{anc.state}:
\begin{enumerate}
\item{\texttt{"ML"}}

Maximum Likelihood - using the \texttt{ace\{ape\}} function
\item{\texttt{"Bayesian"}}

Bayesian method - using the \texttt{anc.Bayes\{phangorn\}} function

\item{\texttt{"Threshold"}}

Threshold method - using the \texttt{threshBayes\{phytools\}} function
\end{enumerate}
Note that only the \texttt{"ML"} option is available for now.

\begin{knitrout}
\definecolor{shadecolor}{rgb}{0.969, 0.969, 0.969}\color{fgcolor}\begin{kframe}
\begin{alltt}
\hlstd{anc.matrix.save}\hlkwb{<-}\hlkwd{anc.state}\hlstd{(tree, table,} \hlkwc{method}\hlstd{=}\hlstr{'ML'}\hlstd{,} \hlkwc{verbose}\hlstd{=}\hlnum{TRUE}\hlstd{)}
\end{alltt}
\end{kframe}
\end{knitrout}

No need to run this function here since we already loaded the ancestral state matrix with the data (~\label{data}).

Let's check what's in \texttt{anc.matrix.save}:

\begin{knitrout}
\definecolor{shadecolor}{rgb}{0.969, 0.969, 0.969}\color{fgcolor}\begin{kframe}
\begin{alltt}
\hlkwd{names}\hlstd{(anc.matrix.save)}
\end{alltt}
\begin{verbatim}
## [1] "state" "prob"
\end{verbatim}
\begin{alltt}
\hlkwd{head}\hlstd{(anc.matrix.save}\hlopt{$}\hlstd{state[,}\hlnum{1}\hlopt{:}\hlnum{5}\hlstd{])} \hlcom{#character 1 to 5 only}
\end{alltt}
\begin{verbatim}
##                [,1] [,2] [,3] [,4] [,5]
## Nanolestes     "0"  "0"  "0"  "?"  "0" 
## Peramus        "0"  "0"  "0"  "?"  "0" 
## Vincelestes    "0"  "0"  "3"  "0"  "0" 
## Kielantherium  "0"  "0"  "0"  "?"  "?" 
## Deltatheridium "0"  "0"  "1"  "0"  "0" 
## Mayulestes     "0"  "0"  "1"  "0"  "0"
\end{verbatim}
\begin{alltt}
\hlkwd{tail}\hlstd{(anc.matrix.save}\hlopt{$}\hlstd{prob[,}\hlnum{1}\hlopt{:}\hlnum{5}\hlstd{])} \hlcom{#character 1 to 5 only}
\end{alltt}
\begin{verbatim}
##           [,1]      [,2]      [,3]      [,4]      [,5]
## n96  0.9999999 0.9999992 0.9909884 0.9923644 0.9586465
## n97  0.9999997 0.9999984 0.9964342 0.9368176 0.9857296
## n98  0.9999976 0.9999886 0.9917309 0.9266584 0.9887148
## n99  0.9999999 0.9998679 0.9722163 0.9902736 0.9388473
## n100 0.9999996 0.9674255 0.5060960 0.9868221 0.5319853
## n101 0.9999998 0.9999957 0.9950809 0.9517160 0.9880845
\end{verbatim}
\end{kframe}
\end{knitrout}

\texttt{anc.matrix.save} is a list of two objects: state containing the states for each character and prob containing the probability of each state to be correct.

\subsection{Probability threshold}
We can then \textit{clean} \texttt{anc.matrix.save} by replacing each character state with a probability below a threshold by missing data (\texttt{"?"}) using the \texttt{anc.unc} function.
Let's use a 0.95 probability threshold (discarding estimation below that):

\begin{knitrout}
\definecolor{shadecolor}{rgb}{0.969, 0.969, 0.969}\color{fgcolor}\begin{kframe}
\begin{alltt}
\hlstd{anc.matrix}\hlkwb{<-}\hlkwd{anc.unc}\hlstd{(anc.matrix.save,} \hlnum{0.95}\hlstd{)}
\end{alltt}
\end{kframe}
\end{knitrout}

\subsection{Calculation the PCO (or MDS) matrix}
We can now run a principal coordinates analysis on \texttt{anc.matrix} using the \texttt{pco.std} function.
This function generates a distance matrix using the \texttt{veg.dist\{vegan\}} function and runs a PCO analysis on it using the \texttt{pcoa\{ape\}} function.
The function takes the options from both \texttt{veg.dist\{vegan\}}, \texttt{pcoa\{ape\}} and \texttt{scale\{base\}}.
Let's run the PCO on a scaled euclidean distance matrix without NAs.

\begin{knitrout}
\definecolor{shadecolor}{rgb}{0.969, 0.969, 0.969}\color{fgcolor}\begin{kframe}
\begin{alltt}
\hlstd{pco}\hlkwb{<-}\hlkwd{pco.std}\hlstd{(anc.matrix,} \hlkwc{distance}\hlstd{=}\hlstr{"euclidean"}\hlstd{,}
    \hlkwc{scale}\hlstd{=}\hlnum{TRUE}\hlstd{,} \hlkwc{center}\hlstd{=}\hlnum{FALSE}\hlstd{,} \hlkwc{na.rm}\hlstd{=}\hlnum{TRUE}\hlstd{,} \hlkwc{correction}\hlstd{=}\hlstr{"none"}\hlstd{)}
\end{alltt}


{\ttfamily\noindent\color{warningcolor}{\#\# Warning in pco.std(anc.matrix, distance = "{}euclidean"{}, scale = TRUE, center = FALSE, : This function is in development.\\\#\# If NA's are introduced in the distance matrix they are replaced by the mean distance.\\\#\# This is methodological wrong!}}\end{kframe}
\end{knitrout}

\section{Analysing the results}
To visualise the results of this PCO, we can use the three plotting options of the \texttt{plot.std} function.

\subsection{Axis analysis}
First let's check the load on each axis in our PCO and the cumulative explicative variance.
We can obtain that by running the \texttt{plot.std} function on the raw \texttt{pco} object.

\begin{knitrout}
\definecolor{shadecolor}{rgb}{0.969, 0.969, 0.969}\color{fgcolor}\begin{kframe}
\begin{alltt}
\hlkwd{plot.std}\hlstd{(pco,} \hlkwc{legend}\hlstd{=}\hlnum{TRUE}\hlstd{)}
\end{alltt}
\end{kframe}

{\centering \includegraphics[width=\maxwidth]{figure/minimal-plot_load-1} 

}



\end{knitrout}

\subsection{Plotting the full character-space}
We can also plot the full character-space to see which group of species occupy which part of the character-space.
First we need to create a \texttt{pco.scores} object using the function with the same name.
This function takes as input our \texttt{tree}, the related \texttt{pco} object and an optional list of taxonomic information.
We already generated the taxonomic list using \texttt{set.data} (~\label{data}).

\begin{knitrout}
\definecolor{shadecolor}{rgb}{0.969, 0.969, 0.969}\color{fgcolor}\begin{kframe}
\begin{alltt}
\hlstd{taxonomy.list}
\end{alltt}
\begin{verbatim}
## $Placental
## [1] 153
## 
## $Stem_placental
## [1] 103 153
\end{verbatim}
\end{kframe}
\end{knitrout}

This list contains the names of the taxonomic groups and the number of the node including them.
Note that the Stem Placental group contains two nodes and is therefore a grade.

\begin{knitrout}
\definecolor{shadecolor}{rgb}{0.969, 0.969, 0.969}\color{fgcolor}\begin{kframe}
\begin{alltt}
\hlstd{pco.scores}\hlkwb{<-}\hlkwd{as.pco.scores}\hlstd{(tree, pco,} \hlkwc{n.axis}\hlstd{=}\hlnum{2}\hlstd{, taxonomy.list)}
\end{alltt}
\end{kframe}
\end{knitrout}

The plot can then be easily created using the \texttt{plot.std} function:

\begin{knitrout}
\definecolor{shadecolor}{rgb}{0.969, 0.969, 0.969}\color{fgcolor}\begin{kframe}
\begin{alltt}
\hlkwd{plot.std}\hlstd{(pco.scores,} \hlkwc{legend}\hlstd{=}\hlnum{TRUE}\hlstd{,} \hlkwc{main}\hlstd{=}\hlstr{"Full character-space"}\hlstd{)}
\end{alltt}
\end{kframe}

{\centering \includegraphics[width=\maxwidth]{figure/minimal-plot_pco-1} 

}



\end{knitrout}

\subsection{Plotting the evolution of the character-space through time}
Finally, we can plot the evolution of this character space through time using the \texttt{slice.std} function.
This function takes as input our \texttt{tree}, the related \texttt{pco.scores} object, the number of slices and a slicing method.
The number of slices is a list of ages that we already generated though \texttt{set.data} (~\label{data}).

\begin{knitrout}
\definecolor{shadecolor}{rgb}{0.969, 0.969, 0.969}\color{fgcolor}\begin{kframe}
\begin{alltt}
\hlstd{slices}
\end{alltt}
\begin{verbatim}
## [1]   0  40  50  60  70  80  90 100 110
\end{verbatim}
\end{kframe}
\end{knitrout}

The \texttt{slice.std} function allows three different tree slicing method:
\begin{enumerate}
\item{\texttt{"ACCTRAN"}}

This method is similar to the parsimony ACCTRAN method and sets the state at the slice to be the one of the closest \textbf{offspring} tip or node.
\item{\texttt{"DELTRAN"}}

This method is similar to the parsimony DELTRAN method and sets the state at the slice to be the one of the closest \textbf{ancestor} tip or node.
\item{\texttt{"RATES"}}

This method allows to give a probability of picking the offspring or the ancestor state depending on the overall character rate and the length of the sliced branch.
\end{enumerate}
Note that the \texttt{"RATES"} option is available for now.

\begin{knitrout}
\definecolor{shadecolor}{rgb}{0.969, 0.969, 0.969}\color{fgcolor}\begin{kframe}
\begin{alltt}
\hlstd{std.slice_acc}\hlkwb{<-}\hlkwd{std.slice}\hlstd{(tree, pco.scores, slices,} \hlkwc{method}\hlstd{=}\hlstr{"ACCTRAN"}\hlstd{)}
\end{alltt}
\end{kframe}
\end{knitrout}

Similarly as above, to plot the results of this function, just use the \texttt{plot.std} function:

\begin{knitrout}
\definecolor{shadecolor}{rgb}{0.969, 0.969, 0.969}\color{fgcolor}\begin{kframe}
\begin{alltt}
\hlkwd{plot.std}\hlstd{(std.slice_acc,} \hlkwc{legend}\hlstd{=}\hlnum{TRUE}\hlstd{,} \hlkwc{pars}\hlstd{=}\hlkwd{c}\hlstd{(}\hlnum{3}\hlstd{,}\hlnum{3}\hlstd{),} \hlkwc{pos.leg}\hlstd{=}\hlkwd{c}\hlstd{(}\hlopt{-}\hlnum{5}\hlstd{,}\hlnum{5}\hlstd{))}
\end{alltt}
\end{kframe}

{\centering \includegraphics[width=\maxwidth]{figure/minimal-plot_slice-1} 

}



\end{knitrout}

And that's it for now!

I'm currently developing the three caveats of this method (and hopefully the will be updated soon):
\begin{enumerate}
\item{The ancestral states estimation:} Namely using Bayesian or the Threshold method to have a better way to deal with uncertainty. One option is also to use other softwares (i.e. MrBayes) to optimize the speed and the results of this step.
\item{The PCO analysis:} Making sure that the PCO analysis is statistically correct, using the right distance matrix, and scaling rules.
\item{The Tree Slicing:} Implementing the \texttt{"RATES"} method to allow to select for the characters states in a less arbitrary way.
\end{enumerate}
And two future directions needs to be developed:
\begin{enumerate}
\item{Adding fossils FOD/LOD values in the analysis.}
\item{Adding body mass to the analysis.}
\item{Adding methods to calculate de disparity using the group variance/range and distance to centroid.}
\end{enumerate}

\end{document}

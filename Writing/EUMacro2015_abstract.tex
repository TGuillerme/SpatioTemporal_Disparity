\documentclass[a4paper,11pt]{article}


\usepackage{natbib}
\usepackage{enumerate}
\usepackage[osf]{mathpazo}
\usepackage{lastpage} 
\pagenumbering{arabic}
\linespread{1.66}

\begin{document}

\begin{flushright}
Version dated: \today
\end{flushright}
\begin{center}

%Title
\noindent{\Large{\bf{Extinction effects on niche occupancy in mammals after the K-Pg boundary}}}\\
\bigskip
%Author
\noindent{Thomas Guillerme and Natalie Cooper \\guillert@tcd.ie - http://tguillerme.github.io/}\\

\end{center}
%\section{Abstract}
Understanding how global biodiversity changes in the context of the sixth mass extinction event is central to macroecology. biotic and biotic changes can have cascading effects on biodiversity and lead to drastic changes in species richness and composition as well as changes in ecological roles and dominant clades. Past mass extinctions have often had a turn-over effect on biodiversity ranging from the extinction of entire clades during the event to the radiation of often unrelated clades after it. One proposed hypothesis is that surviving clades can undergo an adaptive radiation by filling the ecological niches left vacant by the extinct groups.
In this study, we propose a new approach to look at the effects of the K-Pg extinction on morphological niche occupancy in eutherian mammals. We use morphological disparity and phylogenetic diversity as two proxies for estimating niche occupancy alongside Total-Evidence tip-dated trees containing both living and fossil taxa allowing us to apply a novel and finer grained method to analyse disparity and diversity through time.
Our results show that eutherian mammals don't display significant changes in morphological disparity and in diversity after the K-Pg boundary. This suggests that changes in biodiversity after an extinction event can be independent of vacancy in ecological niches triggered by species extinction.

\end{document}

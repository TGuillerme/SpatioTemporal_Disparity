\documentclass[11pt]{letter}
\usepackage[a4paper,left=2.5cm, right=2.5cm, top=1cm, bottom=1cm]{geometry}
\usepackage[osf]{mathpazo}
\signature{Thomas Guillerme \\ Natalie Cooper}
\address{Zoology building \\ Trinity College Dublin \\ Dublin 2, Ireland \\ \\ guillert@tcd.ie}
\longindentation=0pt
\begin{document}

\begin{letter}{}
\opening{Dear Editors,}

The fact that mammals could only diversify after the extinction of the fierce and dominant mesozoic non-avian dinosaurs is a textbook example of faunal replacements and of the long term effect of the Cretaceous-Paleogene (K-Pg) mass extinction.
However, this example has been heavily debate in the last four years (Meredith et al 2011 Science; O'Leary et al 2013 Science; Springer et al 2013 Science; dos Reis et al 2014 Biology Letters) with mixed evidences supporting a diversification of mammals post-K-Pg (suggesting an effect of the extinction of non-avian dinosaurs) or pre-K-Pg (suggesting no effect of the extinction event).
A major part of this debates relies in different signal from different sources of data (palaeontological data \textit{vs.} neontological data) as well as the use of taxonomic diversity as a proxy for measuring mammalian diversification.

In this research article, entitled ``Mammalian morphological diversity does not increase in response to the Cretaceous-Paleogene mass extinction and the extinction of the (non-avian) dinosaurs'', we propose a resolution of this debate by using both palaeontological and neontological data through the use of Total Evidence tip-dated phylogenies (from Slater 2013 Methods Ecol. Evol. and Beck \& Lee 2014 Proceedings Roy. Soc. B) as well as using morphological diversity (i.e. disparity) rather than taxonomic diversity to estimate morphological diversification.
To our knowledge, our article is the first to propose such an approach to the mammal diversification debate.

By using data from both living and fossil taxa and a disparity through time approach, we found no evidences for a direct effect of the K-Pg extinction event on mammalian diversification.
We therefore propose, contrary to popular belief, that the extinction of non-avian dinosaurs at the K-Pg boundary, 66 million years ago, was not the factor that allowed mammalians to diversify during the Tertiary.

We look forward to hearing from you soon,

\closing{Yours sincerely,}

\end{letter}
\end{document}

% Suggested reviewers
%Graeme Lloyd (remove from acknowledgements)
%Robin Beck


% In recent years there has been growing interest in building phylogenies that contain both living and fossil taxa (e.g. Quental and Marshall 2006 TREE; Fritz et al. 2013 TREE; Heath et al. 2014 PNAS). Such phylogenies could revolutionize the way we think about macroevolutionary patterns and processes, and provide a more complete understanding of trends in biodiversity through time. Unfortunately building such trees has proved technically difficult. 

% One method, the Total Evidence method, allows us to use molecular and morphological data to build phylogenies with both living and fossil species as tips (Ronquist et al. 2012 Syst Biol). This method is extremely promising because it allows us to use all the available data. However, because of the amount of data involved, the Total Evidence method is likely to be affected by missing data.

% Our research article, entitled ``Effects of missing data on topological inference using a Total Evidence approach'', is to our knowledge, the first to thoroughly analyze the effects of missing data on tree topology in a Total Evidence framework. Using simulations ($>$ 150 CPU years worth), we find that the number of living taxa with morphological data and the overall number of morphological characters, are more important than the amount of missing data in the fossil record for recovering the ``best'' tree topology. Additionally, we show that Bayesian methods outperform Maximum Likelihood methods, regardless of the amount of missing data.

% Our results suggest that increasing the number of taxa with morphological data and the overall number of morphological characters will greatly improve the quality of Total Evidence tree topologies. This has major implications for clades where detailed morphological data collection from living species is rare. Additionally, we recommend using a Bayesian majority consensus tree when fixing tree topology for any additional analyses.

% We look forward to hearing from you soon,

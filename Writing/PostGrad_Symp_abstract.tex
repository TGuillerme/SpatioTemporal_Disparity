\documentclass[a4paper,11pt]{article}


\usepackage{natbib}
\usepackage{enumerate}
\usepackage[osf]{mathpazo}
\usepackage{lastpage} 
\pagenumbering{arabic}
\linespread{1.66}

\begin{document}

\begin{flushright}
Version dated: \today
\end{flushright}
\begin{center}

%Title
\noindent{\Large{\bf{Tempo and mode in mammals morphological evolution around the K-Pg boundary}}}\\
\bigskip
%Author
\noindent{Thomas Guillerme - guillert@tcd.ie - http://tguillerme.github.io/}\\

\end{center}

\section{Abstract}
The Cretaceous-Paleogene boundary (K-T; 66Mya) represents a drastic global change in biodiversity. This event is linked to an extraterrestrial impact and a major volcanism event. Traditionally the K-T event was viewed as the extinction of dinosaurs and the rise of mammals. However, the last three decades of field palaeontology and macroevolutionary studies have weakened this simplistic view. Exceptionally diverse mammals appeared prior to the Paleogene and there is overwhelming evidence for a radiation of the avian-dinosaurs in the latest Cretaceous.

In this study, we use a character space approach for a data set of living and fossil mammals to investigate the variation in disparity and diversity in mammals since the Jurassic. We show that maximal disparity is achieved prior to the Paleogene and that there is no significant increase in diversity across the K-T boundary.

\end{document}

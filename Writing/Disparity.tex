\documentclass[12pt,letterpaper]{article}

%Packages
\usepackage{pdflscape}
\usepackage{fixltx2e}
\usepackage{textcomp}
\usepackage{fullpage}
\usepackage{float}
\usepackage{latexsym}
\usepackage{url}
\usepackage{epsfig}
\usepackage{graphicx}
\usepackage{amssymb}
\usepackage{amsmath}
\usepackage{bm}
\usepackage{array}
\usepackage[version=3]{mhchem}
\usepackage{ifthen}
\usepackage{caption}
\usepackage{hyperref}
\usepackage{amsthm}
\usepackage{amstext}
\usepackage{enumerate}
\usepackage[osf]{mathpazo}
\usepackage{dcolumn}
\usepackage{lineno}
\pagenumbering{arabic}

%Pagination style and stuff
\linespread{2}
\raggedright
\setlength{\parindent}{0.5in}
\setcounter{secnumdepth}{0} 
\renewcommand{\section}[1]{%
\bigskip
\begin{center}
\begin{Large}
\normalfont\scshape #1
\medskip
\end{Large}
\end{center}}
\renewcommand{\subsection}[1]{%
\bigskip
\begin{center}
\begin{large}
\normalfont\itshape #1
\end{large}
\end{center}}
\renewcommand{\subsubsection}[1]{%
\vspace{2ex}
\noindent
\textit{#1.}---}
\renewcommand{\tableofcontents}{}

\begin{document}

\section{Problems with calculating disparity}

\begin{enumerate}
\item{Removing the last PC axis: no real valid justification}
\item{Product of variance (and of ranges?)}
\item{Sum of variance (and of ranges?)}
\end{enumerate}

\subsection{Product of variance}

The product of variance is calculated classically \cite{Wills1994} as follow for a $n$ dimension volume (PCO/PCA):
\begin{equation}
\prod\limits_{X=1}^n \sigma^2 = \sigma_{X1}^2 \times \sigma_{X2}^2 \times ... \times \sigma_{Xn}^2
\end{equation}
However, this is only correct if each axis of the $n$ object are independent, which is by definition not the case for a $n$ dimension volume from a PCO/PCA because the matrix is transformed using the covariance matrix. %is that right?

The problem with these methods is that they assume the data is multivariate normal and independent. Both assumptions are easily violated. In this paper we propose a new view of the whole thing. Where we can measure the volume and the spread within the $n$ dimension space.

For example, the eigen decomposition of a distance matrix containing three taxa ($k=3$) creates a cladisto-space of $n=2$ dimensions.
In this case, the product of variance, $\prod\sigma^2$, of this cladisto-space is:
\begin{equation} % NC: Check and double check this is the correct equation.
   \prod\sigma^2=
   \begin{cases}
     \sigma^2(n_1) \times \sigma^2(n_2), & \text{if}\ \sigma(n_1,n_2)=0 \\
     [\sigma^2(n_1)+2\sigma(n_1,n_2)] \times [\sigma^2(n_2)+2\sigma(n_1,n_2)], & \text{otherwise}
   \end{cases}
   \label{prod_var}
\end{equation}
Where $\sigma^2(n)$ is the variance of the eigenvector $n$ and $\sigma(n_{1}, n_{2})$ is the covariance between the two eigenvectors.
Note that the same is true for the sum of variance with the difference that the terms are added rather than multiplied (equation \ref{prod_var}).


Two other ways to calculate the variation in the data is: (1) the distance from centroid \cite{finlay2015morphological} representing the spread of the data in the $n$ dimension volume and (2) the $n$ dimension ellipsoid volume \cite{DonohueDim} representing the occupancy of the data in the $n$ dimension volume.

1-The average distance from centroid is defined as follow:
\begin{equation}
Distance=\frac{{\sum\limits_{i=1}^k \sqrt{(x_{in}-\bar{x}_{n})^2}}}{k}
\end{equation}

Where $k$ is the size of the distance matrix; $n$ us the number of dimensions; $x_{in}$ are all the values observed in the $n^th$ dimension and $\bar{x}_{n}$ is the mean value observed in the $n^th$ dimension.
%\begin{equation}
%\bar{x_n}={\sum\limits_{i=1}^k(x_{in})}/k
%\end{equation}

2-The $n$ dimension ellipsoid volume is defined as follow:
\begin{equation}
Volume={{\pi^{n/2}} \over {\Gamma({n\over2}+1)}}\prod\limits{i=1}^n(\lambda_{i}^{0.5})
\end{equation}
Where $n$ is the number of dimensions; $\Gamma$ is a gamma distribution and $\lambda_i$ is the $i^{th}$ eigen value of the covariance matrix.1.


\section{Problems with calculating disparity through time}

\begin{enumerate}
\item{Using bio-stratigraphy for time bins}
\end{enumerate}

\subsection{Using bio-stratigraphy for time bins}

It's baaaad.

\subsection{New time slicing}

It's better!

\begin{table}[ht]
\caption{Comparison of Cladisto-space studies methods}
\centering
\begin{tabular}{cccccccc}
  \hline
    Date & Author      & Distance  & Ordination & Binning    & Disparity   & Difference & cite \\ %
  \hline
         & this study  & Gower     & PCO        & Time slice & centroid    & NPMANOVA?  & \\
    2014 & Benson      &           &            & Equal bins & Wills 1994* & NPMANOVA   & \cite{bensonfaunal2014} \\
    2014 & Brusatte    & Euclidean & PCO        &            &             &            & \cite{brusattegradual2014} \\
    2014 & Benton      & Euclidean & PCO        & Biostrat   & Wills 1994* & NPMANOVA   & \cite{bentonmodels2014} \\
    2013 & Hopkins     &           &            & Equal bins & Wills 1994* &            & \cite{hopkinsdecoupling2013} \\             
    2013 & Ruta        & GED       & PCO        & Biostrat   & Wills 1994* & NPMANOVA   & \cite{ruta2013} \\
    2013 & Hughes      & Euclidean & PCO        & Biostrat   & Sum of var  &            & \cite{Hughes20082013} \\
    2013 & Toljagic    & Euclidean & PCO        & Biostrat   & Wills 1994* & NPMANOVA   & \cite{toljagictriassic-jurassic2013} \\
    2012 & Brusatte    & Euclidean & PCO        & Biostrat   & Wills 1994* & CI overlap & \cite{brusattedinosaur2012} \\
    2012 & Anderson    & Gower     & PCO        &            &             &            & \cite{anderson2012using} \\
    2010 & Prentice    & Euclidean & PCO        & Biostrat   & Wills 1994* & NPMANOVA   & \cite{prentice2011} \\
    2011 & Thorne      & Euclidean & PCO        & Biostrat   &             & NPMANOVA   & \cite{thorneresetting2011} \\
    2010 & Cisneros    & Euclidean & PCO        & Biostrat   & Wills 1994* & NPMANOVA   & \cite{cisneros2010} \\
    2008 & Brusatte    & Euclidean & PCO        & Biostrat   & Wills 1994* & NPMANOVA   & \cite{brusatte50} \\
    2008 & Brusatte    & Euclidean & PCO        & Biostrat   & Wills 1994* & NPMANOVA   & \cite{Brusatte12092008} \\
    2005 & Wesley-Hunt &           & PCO        &            & Foote 1992  & t-test     & \cite{Wesley-Hunt2005} \\
  \hline
\end{tabular}
\end{table}
* The 4 sum and product of range and variance



Good review in Ciampaglio.et.al~2001-Paleobiology.

\begin{landscape}
\begin{table}[H]
\caption{Disparity metrics descriptions from Wills et al 1994}
\centering
\begin{tabular}{p{4.5cm}p{2cm}p{1cm}p{10cm}p{6cm}}
  \hline
    Type of measurement & Type(II) & $N^o$ & Name & Description \\ %
  \hline
  Morphological variety & Volumes   & 1  & Total number of different character states       & Character-State Variability as an Index of Morphological Variety \\
                        &           & 2  & Product of variances                             & average dissimilarity among forms \\
                        &           & 3  & Product of ranges                                & overall morphological variation\\
                        &           & 4  & Hyper-ellipsoid volumes                          & \\
                        &           & 5  & Sum of variances                                 & \\
                        &           & 6  & Sum of ranges                                    & \\
  Distance measures     & Phenetic  & 7  & Mean euclidean distance from group centroid      & \\
                        &           & 8  & Mean euclidean distance from overall centroid    & \\
                        &           & 9  & Mean euclidean distance from generalized annelid & \\
                        &           & 10 & Mean euclidean distance between all taxa         & \\
                        &           & 11 & Mean Manhattan distance between all taxa         & \\
                        & Cladistic & 12 & Mean patristic distance from generalized annelid & \\
                        &           & 13 & Mean patristic distance between all taxa         & \\  
                        &           & 14 & (b-a)/a                                          & \\                          
  \hline
\end{tabular}
\end{table}
\end{landscape}

\begin{landscape}
\begin{table}[H]
\caption{Disparity metrics definitions from Wills et al 1994}
\centering
\begin{tabular}{p{1cm}p{5cm}p{10cm}p{10cm}}
  \hline
    $N^o$ & Name & Calculation & Terms \\ %
  \hline
  1  & Total number of different character states       & \(\sum_{i=1}^{N}{(S_{i}-1)}\) & $S$ being the number of states manifested by the subset of taxa for each character \\
  2  & Product of variances                             & \(\prod{n=1}^{N}{var{n}}\) & $n$ being each dimension \\
  3  & Product of ranges                                & \(\prod{n=1}^{N}{(max{n}-min{n})}\) & $n$ being each dimension \\
  4  & Hyper-ellipsoid volumes                          & & \\ %Wills p 8
  5  & Sum of variances                                 & & \\
  6  & Sum of ranges                                    & & \\
  7  & Mean euclidean distance from group centroid      & & \\
  8  & Mean euclidean distance from overall centroid    & & \\
  9  & Mean euclidean distance from generalized annelid & & \\
  10 & Mean euclidean distance between all taxa         & & \\
  11 & Mean Manhattan distance between all taxa         & & \\
  12 & Mean patristic distance from generalized annelid & & \\
  13 & Mean patristic distance between all taxa         & & \\  
  14 & (b-a)/a                                          & & \\                          
  \hline
\end{tabular}
\end{table}
\end{landscape}


\bibliographystyle{vancouver}
\bibliography{References}

%END
\end{document}


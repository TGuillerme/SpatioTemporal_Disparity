\documentclass[11pt]{letter}
\usepackage[a4paper,left=2.5cm, right=2.5cm, top=1cm, bottom=1cm]{geometry}
\usepackage[osf]{mathpazo}
\usepackage{url}
\signature{Thomas Guillerme \\ Natalie Cooper}
\address{School of Natural Sciences \\ Trinity College Dublin \\ Dublin 2, Ireland \\ guillert@tcd.ie}
\longindentation=0pt
\begin{document}



% PLoS Biology cover letter:
% Upload a cover letter as a separate file in the online system. The length limit is 600 words.

% The cover letter should address the following questions:

%     What is the scientific question you are addressing?
% "investigate whether mammal diversity was directly affected by the K-Pg mass extinction."
%     What is the key finding that answers this question?
% "We find no evidence for a direct effect of the K-Pg extinction event on mammalian diversification."
%     What is the nature of the evidence you provide in support of your conclusion?
% The time slicing method and using both data?
%     What are the three most recently published articles that are relevant to this question?
% Slater 2013 Methods Ecol. Evol.; Beck \& Lee 2014 Proc. Roy. Soc. B Close et al. 2015 Current Biol. (Or maybe put O'Leary?)
%     What significance do your results have for the field?
% Proposing new methods to investigate disparity-through-time in a continuous way + using both living and fossil taxa
%     What significance do your results have for the broader community (of biologists and/or the public)?
% Text book example.
%     What other novel findings do you present?
% Effect of using both data sources
%     Is there additional information that we should take into account?
% Graeme's story?


% TG: Ok, nothing too complex here, just some reformatting maybe. Everything is clearly stated in the Nature Comm version but not in their order. Maybe do a bullet point alternative version?

\begin{letter}{}
\opening{Dear Editors,}

The idea that mammals could only diversify after the extinction of the fierce and dominant non-avian dinosaurs, is a textbook example of faunal replacements and of the long-term effects of the Cretaceous-Paleogene (K-Pg) mass extinction.
However, this example has been heavily debated in the last four years (Meredith et al 2011 Science; O'Leary et al 2013 Science; Springer et al 2013 Science; dos Reis et al 2014 Biology Letters), with results seeming to depend on whether fossil or molecular data were used.
Analyses of fossil data support a rapid diversification of mammals soon after the K-Pg boundary (suggesting an effect of the mass extinction event) whereas molecular data finds no such effect.
If we are ever to solve this fundamental debate we need to include both fossil and living taxa, and explore more aspects of diversity than taxonomic richness.

In this research article, entitled ``Investigating the effects of the Cretaceous-Paleogene mass extinction on mammalian morphological diversity using a new methodological approach'', we use both data from living and fossil species through Total Evidence tip-dated phylogenies along with a novel continuous time-slicing method to investigate whether mammalian morphological diversity was directly affected by the K-Pg event.

We used all the available data (palaeontological and neontological) for two independent datasets (Mammaliaformes and Eutheria - available at \url{http://dx.doi.org/10.6084/m9.figshare.1539545
}) alongside cutting-edge methods, including a novel time-slicing approach to estimate disparity-through-time.
We find that mammalian disparity just before K-Pg is not significantly different than at any time-slice after K-Pg, using two independent datasets.
We therefore propose that the extinction of many terrestrial vertebrates at the K-Pg boundary, did not affect the diversification of mammals during the Tertiary. 
% We find that mammalian morphological diversity (disparity) just before K-Pg is not significantly different than at any time-slice after K-Pg, using two independent datasets (Mammaliaformes families and Eutheria genera).
% Our results were generated using all the available data (palaeontological and neontological) and cutting-edge methods, including a novel time-slicing approach to estimate disparity-through-time, and using disparity rather than taxonomic diversity as our proxy for diversity.
% We therefore propose that the extinction of many terrestrial vertebrates at the K-Pg boundary, was not a major factor in the diversification of mammals during the Tertiary. 
%This question has recently been investigated by looking at body-mass evolution (Slater 2013 Methods Ecol. Evol.), diversification rates (Beck \& Lee 2014 Proc. Roy. Soc. B) and disparity-through-time prior to K-Pg (Close et al. 2015 Current Biol.).
This is the first study, to our knowledge, to apply state-of-the-art disparity-through-time analyses to both living and fossil species to investigate this question.

These results provide a solution to the debate about whether mammals were affected by the extinction of the non-avian dinosaurs.
More broadly, our time-slicing approach will be useful for future disparity-through-time analyses because it avoids some major caveats of previously used methods.
%An \texttt{R} package is also currently being developed to facilitate users to reproduce or use this method (\url{https://github.com/TGuillerme/dispRity}). 
% NC: My comment would be, ok why not provide it now?

%We hope that our findings will also shed a new light on the complexity of mammalian evolution (and therefore our own) and that these results will be of interest both to the broader scientific community, and to the public as they involve early mammals and dinosaurs, two groups that tend to capture the public imagination. 
% NC: This is a bit of a stretch, and probably not needed for Evolution.

% NC: This is fine as part of the letter, postscripts are always a bit annoying.
Note that we have included Graeme Lloyd as a suggested reviewer even though he is mentioned in the acknowledgments of the article.
Graeme is a recognised expert in disparity-through-time analyses (e.g. Close, Friedman, Lloyd and Benson, 2015 Current Biology) so we think he would be an ideal reviewer.
However, we had some interactions with Graeme in the early stages of the project while preparing code for the analyses hence he is acknowledged.
We wanted to state this upfront to allow editors to determine whether this forms a conflict of interest.

We look forward to hearing from you,

\closing{Yours sincerely,}


\end{letter}
\end{document}

% Editors
% Dean Adams?
% Matt Friedman?
% Daniel Rabosky?
% Liam Revell?

% Suggested reviewers:
% Graeme Lloyd
% Robin Beck
% Robert Close
% Mike Benton or Steve Brusatte or Richard Butler (they're cited a lot)
% Any other ideas?

% Yep all good.
% BUT if you want to suggest Graeme you need to add the postscript I added above. You may need to alter the reference.



\documentclass[11pt]{letter}
\usepackage[a4paper,left=2.5cm, right=2.5cm, top=1cm, bottom=1cm]{geometry}
\usepackage[osf]{mathpazo}
\signature{Thomas Guillerme \\ Natalie Cooper}
\address{Zoology building \\ Trinity College Dublin \\ Dublin 2, Ireland \\ \\ guillert@tcd.ie}
\longindentation=0pt
\begin{document}

\begin{letter}{}
\opening{Dear Editors,}

The idea that mammals could only diversify after the extinction of the fierce and dominant non-avian dinosaurs, is a textbook example of faunal replacements and of the long-term effects of the Cretaceous-Paleogene (K-Pg) mass extinction.
However, this example has been heavily debated in the last four years (Meredith et al 2011 Science; O'Leary et al 2013 Science; Springer et al 2013 Science; dos Reis et al 2014 Biology Letters), with results seeming to depend on whether fossil or molecular data were used. 
Analyses of fossil data find that mammals diversified rapidly soon after the K-Pg boundary, suggesting an effect of the extinction of non-avian dinosaurs on mammalian evolution. 
Analyses of molecular data from living species however, find no effect of the mass extinction on mammalian diversification. 
If we are ever to solve this fundamental debate we need studies that include both fossil and living taxa, and explore more aspects of diversity than taxonomic richness.
%A major source of this debates comes from the different signal from different sources of data (palaeontological data \textit{vs.} neontological data) as well as the use of taxonomic diversity as a proxy for measuring mammalian diversification.

In this research article, entitled ``Mammalian morphological diversity does not increase in response to the Cretaceous-Paleogene mass extinction and the extinction of the (non-avian) dinosaurs'', we use both palaeontological and neontological data through Total Evidence tip-dated phylogenies (from Slater 2013 Methods Ecol. Evol. and Beck \& Lee 2014 Proc. Roy. Soc. B) to investigate whether mammal diversity was directly affected by the K-Pg mass extinction. We use cutting-edge methods, including a novel time-slicing approach to estimate disparity-through-time, and use morphological diversity (disparity) rather than taxonomic diversity as our proxy for mammalian diversity.
To our knowledge, our article is the first to take such an approach to the debate about the timing of mammal diversification.

We find no evidence for a direct effect of the K-Pg extinction event on mammalian diversification.
We therefore propose that, contrary to popular belief, the extinction of non-avian dinosaurs at the K-Pg boundary, 66 million years ago, was not a major factor in the diversification of mammals during the Tertiary. 

We look forward to hearing from you,

\closing{Yours sincerely,}

\ps{\textit{P.S. Note that we have included Graeme Lloyd as a suggested reviewer even though he is mentioned in the acknowledgments of the article. Graeme is a recognised expert in disparity-through-time analyses (e.g. Close, Freidman, Lloyd and Benson, 2015 Current Biology) so we think he would be an ideal reviewer. However, we had some interactions with Graeme in the early stages of the project while preparing code for the analyses hence he is acknowledged. We wanted to state this upfront to allow editors to determine whether this forms a conflict of interest.}}

\end{letter}
\end{document}

% Editors
% Vera Domingues? (the only one in ecology evolution it seems) http://www.nature.com/ncomms/about_eds/index.html


% Suggested reviewers:
% Graeme Lloyd
% Robin Beck
% Robert Close
% Mike Benton or Steve Brusatte or Richard Butler (they're cited a lot)
% Any other ideas?

% Yep all good.
% BUT if you want to suggest Graeme you need to add the postscript I added above. You may need to alter the reference.


